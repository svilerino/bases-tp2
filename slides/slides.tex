\documentclass{beamer}
\usepackage{graphicx}
\usepackage[utf8]{inputenc}
\usetheme{Warsaw}

\beamertemplatenavigationsymbolsempty 

\title{Bases de Datos No-SQL - Cassandra}
\author{Chapresto, Garassino, Vileriño}
\date{10 de julio de 2015}

\begin{document}

\begin{frame}
  \maketitle
\end{frame}

\begin{frame}
  \frametitle{Intro a No-SQL}    
  \framesubtitle{Porque No-SQL?}
  \begin{itemize}
    \setlength{\itemsep}{3pt}
    \item Crecimiento exponencial de dispositivos conectados a internet
    \pause
    \item Enorme cantidad de nuevos servicios prestados en la nube.
    \pause
    \item Necesidad de escalar rapidamente ante el crecimiento de la demanda.
    \pause
    \item Necesidad de alta disponibilidad, sin downtimes.
    \pause
    \item Bases de datos relacionales no funcionan del todo bien en estos contextos.
  \end{itemize}
\end{frame}

\begin{frame}
  \frametitle{Intro a No-SQL}    
  \framesubtitle{Bases de datos relacionales en problemas}
  \begin{itemize}
    \setlength{\itemsep}{3pt}
    \item Las bd relacionales son buenas para la modificacion de datos manteniendo la consistencia. 
    \pause
    \item Respetan las propiedades llamadas ACID.
    \pause
    \item Las propiedades ACID son deseables pero no siempre necesarias bajo ciertos contextos.
    \pause
    \item Cuando se comenzaron a utilizar sistemas distribuidos por los grandes volumenes de datos se not\'o que los motores relacionales no lograban desarrollar todo su potencial.
    \pause
    \item Teorema CAP: Solo se pueden tener 2 de las 3 caracteristicas: Consistencia, Disponibilidad, Tolerancia a la particion. 
  \end{itemize}
\end{frame}

\begin{frame}
  \frametitle{Intro a No-SQL}    
  \framesubtitle{Bases de datos No SQL}
  \begin{itemize}
    \setlength{\itemsep}{3pt}
    \item Surgen como solucion al problema de escalar horizontalmente
    \pause
    \item Se sacrifica en cierto grado alguna de las propiedades tradicionales(ie. Consistencia) en favor de la capacidad de escalar horizontalmente.
    \pause
    \item Se puede escalar agregando nuevos nodos \texttt{on the fly} sin perder disponibilidad(No downtime).
    \pause
    \item Principio de consistencia eventual: A pesar de que la base de datos pueda ser momentaneamente inconsistente, el estado del sistema se modificara de manera autonoma hasta converger a la consistencia.
  \end{itemize}
\end{frame}

\begin{frame}
  \frametitle{Cassandra}
  \framesubtitle{En que contextos usar Cassandra}
  \begin{itemize}
    \setlength{\itemsep}{3pt}
    \item \textbf{Arquitectura descentralizada:} No existe SPOF. Los nodos se comunican con protocolos p2p. Cada nodo puede satisfacer cualquier request.
    \pause
    \item \textbf{Replicacion a lo largo de varios data-centers: } Cassandra esta pensado para ser desplegado en un gran numero de nodos, manteniendo redundancia de datos para proveer alta disponibilidad.
    \pause
    \item \textbf{Escalable y performante: } El throughput de lecturas y escrituras aumenta linealmente con la cantidad de nodos en el sistema.
  \end{itemize}
\end{frame}

\begin{frame}
  \frametitle{Cassandra}
  \framesubtitle{En que contextos usar Cassandra}
  \begin{itemize}
    \setlength{\itemsep}{3pt}
    \item \textbf{Tolerante a fallas: } La replicacion en multiples nodos brinda resistencia ante la caida de alguno de ellos. Los nodos que fallan son reemplazables sin downtime.
    \pause
    \item \textbf{Multiples niveles de consistencia: } Lecturas y escrituras ofrecen diferentes niveles de consistencia, desde ``las escrituras jamas fallan'' hasta ``bloquear lecturas hasta tener todas las replicas'', con mecanismos de quorum \footnote{ \href{https://en.wikipedia.org/wiki/Quorum_(distributed_computing)}{Tecnicas basadas en quorum para sistemas distribuidos} Wikipedia} como punto intermedio.
    \pause
    \item \textbf{Interface legacy: } Cassandra introduce el lenguaje CQL, parecido a SQL para realizar operaciones \textbf{permitidas}(Recordar que no se pueden hacer joins por ejemplo.).
  \end{itemize}
\end{frame}


\end{document}